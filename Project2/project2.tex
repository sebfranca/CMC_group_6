\documentclass{cmc}
\usepackage{makecell}
\begin{document}

\pagestyle{fancy}
\lhead{\textit{\textbf{Computational Motor Control, Spring 2022} \\
    Final project, Project 2, GRADED}} \rhead{Student \\ Names}

\section*{Student names: \ldots (please update)}

\textit{Instructions: Update this file (or recreate a similar one, e.g.\ in
  Word) to prepare your answers to the questions. Feel free to add text,
  equations and figures as needed. Hand-written notes, e.g.\ for the development
  of equations, can also be included e.g.\ as pictures (from your cell phone or
  from a scanner).  \textbf{\corr{This lab is graded.}} and needs to be
  submitted before the \textbf{\corr{Deadline : Friday 10-06-2022 23:59. For
      project 2, you must submit one final report for all of the following
      exercises separately from the report of project 1. The code of both
      projects can be provided together.}}  Please submit both the source file
  (*.doc/*.tex) and a pdf of your document, as well as all the used and updated
  Python functions in a single zipped file called
  \corr{final\_report\_name1\_name2\_name3.zip} where name\# are the team
  member’s last names.  \corr{Please submit only one report per team!}}
\\

\section*{Questions}

%%%%%%%%%%%%%%%%%%%%%%%%%%%%%%%%%%%%%%%%%%%%%%%%%%%%%%%%%%%%%%%%%%%%%%%%%%%%%%%%%%%%%%%%%%%%%%%%%%%%

\subsection*{9a. Limb – Spine coordination}
\label{sec:limb-spine-coordination}

In this next part you will explore the importance of a proper coordination
between the spine and the limb movement for walking.

\begin{enumerate}
\item Change the drive to a value used for walking and verify that the robot
  walks
\item Analyze the spine movement: What are your phase lags along the spine
  during walking? How does the spine movement compare to the one used for
  swimming?
\item Notice that the phase between limb and spine oscillators affects the
  robot’s walking speed. Run a parameter sweep on the phase offset between
  limbs and spine. Include plots showing how the phase offset influences walking
  speed and comment the results. How do your findings compare to body
  deformations in the salamander while walking?
\item Explore the influence of the oscillation amplitude along the body with
  respect to the walking speed of the robot. Run a parameter search on the
  nominal radius R with a fixed phase offset between limbs and the spine. For
  the phase offset take the optimal value from the previous sub-exercise. While
  exploring R, start from 0 (no body bending).
\end{enumerate}

Include plots showing how the oscillation radius influences walking speed and
comment on the results.



\subsection*{9b. Land-to-water transitions}

\begin{enumerate}
\item In this exercise you will explore the gait switching mechanism. The gait
  switching is generated by a high level drive signal which interacts with the
  saturation functions that you should have implemented in 8a. Implement a new
  experiment which uses the x-coordinate of the robot in the world retrieved
  from the GPS sensor reading (Check \corr{simulation.py} for an example on how
  to access the gps data). Based on the GPS reading,
  you should determine if the robot should walk (it’s on land) or swim (it
  reached water). Depending on the current position of the robot, you should
  modify the drive such that it switches gait appropriately.
\item Run the MuJoCo simulation and report spine and limb angles, together with
  the x coordinate from the GPS signal. Record a video showing the transition
  from land to water and submit the video together with this report.
\item Achieve water-to-land transition. Report spine and limb angles,
  the x-coordinate of the GPS and record a video.
\end{enumerate}


\textbf{Hint:} Use the record options as shown in \corr{exercise\_9b.py} to
easily record videos.



\section*{}
\subsection*{9c. Reserach proposal}

Propose a potential additional study that could be performed
  in simulation and with the real salamander.  This should be written
  like a research proposal using the questions listed below and should not exceed
  2 pages (including figures and references). You are free to choose
  any topic related to sensorimotor coordination and locomotion of the
  salamander. \corr{NOTE : The proposal should be just text (possibly with some
  figures), there is no need to perform the actual numerical
  experiments!}

\begin{enumerate}
\item Provide a scientific question
\item Formulate a hypothesis corresponding to the scientific question
\item Describe an experiment in simulation that could be
  performed to test the hypothesis
\item Specify which type of simulation (e.g. neural circuits +
  biomechanics, neural circuits alone, etc.), which level of
  abstraction, and which assumptions (cf the modeling steps presented
  in the course)
\item Specify a corresponding experiment that could be performed
  with the real animal
\item Discuss what you expect and what could be learned from
  those experiments (in simulation and real)
\item Refer to and include a short bibliography with relevant
  literature (Example \cite{ijspeert2007swimming})
\end{enumerate}

%%%%%%%%%%%%%%%%%%%%%%%%%%%%%%%%%%%%%%%%%%%%%%%%%%%%%%%%%%%%%%%%%%%%%%%%%%%%%%%%%%%%%%%%%%%%%%%%%%%%

% \newpage

\bibliographystyle{ieeetr}
\bibliography{project2}
\label{sec:references}

% \newpage

% \section*{APPENDIX}
% \label{sec:appendix}

\end{document}

%%% Local Variables:
%%% mode: latex
%%% TeX-master: t
%%% End: